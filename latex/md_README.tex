\#\+Buczak Krzysztof Adam

Mój projekt to symulacja mikroskopijnego państwa. Państwo to ma kilka determinujących jego stan współczynników, głównie związanych z gospodarką i światopoglądem (stosunkiem do ważnych dla obywateli spraw). Obywatele to osoby żyjące w państwie, płacące podatki i mające swój własny światopogląd oraz poziom zadowolenia, który jest tym niższy/wyższy, im mniej/bardziej światopogląd państwa odpowiada światopoglądowi danej jednostki i decyduje o produktywności obywatela oraz o tym, czy płaci podatki. Specjalnym rodzajem obywatela jest polityk, który oprócz cech standardowych jest określany także przez charyzmę i prawdomówność.

Co cztery lata (czyli co cztery tury) w państwie dochodzi do powszechnych wyborów prezydenckich, podczas których obywatele decydują, na którego polityka oddać głos. Istotną rolę odgrywa tutaj światopogląd danego polityka (obywatele chętniej zagłosują na kogoś, kto podziela ich światopogląd) i jego charyzma. Wybrany na urząd polityk, w zależności od poziomu prawdomówności, spełnia (bądź nie) swoje postulaty, zmieniając tym samym częściowo światopogląd państwa. Na światopogląd składają się między innymi stosunek państwa do gospodarki, swobód obywatelskich i wysokości podatków. Niektóre składowe światopoglądu mają wymierny wpływ na funkcjonowanie państwa; przykładowo nacisk na gospodarkę kapitalistyczną zwiększy dochody obywateli, podczas gdy socjalistyczna zwiększy ich zadowolenie (oczywiście jest do dość umowna konwencja, ale tego typu detale planuję doszlifować gdy będę mógł zobaczyć działanie symulacji w praktyce i odpowiednio wyważyć poszczególne współczynniki i modyfikatory tak, aby odpowiadały rzeczywistości). Stan początkowy państwa (budżet, wysokość podatków, urzędujący prezydent, światopogląd itd.) jest wybierany losowo. 